\documentclass[UTF8]{ctexart}
\usepackage{graphicx}
\usepackage{listings}
\usepackage{hyperref}
\usepackage{xcolor}
\usepackage{amsmath}
\newtheorem{corollary}{Corollary}[section]
\usepackage{amsmath}

\newtheorem{definition}{Definition}
\newtheorem{theorem}{Theorem}

\title{函数的极限}
\author{seeker}
\date{\today}	

\lstset{
	basicstyle          =   \sffamily,          % 基本代码风格
	keywordstyle        =   \bfseries,          % 关键字风格
	commentstyle        =   \rmfamily\itshape,  % 注释的风格,斜体
	stringstyle         =   \ttfamily,  % 字符串风格
	flexiblecolumns,                % 别问为什么,加上这个
	numbers             =   left,   % 行号的位置在左边
	showspaces          =   false,  % 是否显示空格,显示了有点乱,所以不现实了
	numberstyle         =   \zihao{-5}\ttfamily,    % 行号的样式,小五号,tt等宽字体
	showstringspaces    =   false,
	captionpos          =   t,      % 这段代码的名字所呈现的位置,t指的是top上面
	frame               =   lrtb,   % 显示边框
}

\lstdefinestyle{cpp}{
	language        =   C++, % 语言选cpp
	basicstyle      =   \zihao{-5}\ttfamily,
	numberstyle     =   \zihao{-5}\ttfamily,
	breaklines      =   true,   % 自动换行,建议不要写太长的行
	columns         =   fixed,  % 如果不加这一句,字间距就不固定,很丑,必须加
	basewidth       =   0.5em,
}

\begin{document}
	\maketitle
	\section{自变量趋于无穷大时}
	\begin{definition}
		$\lim _{x \rightarrow \infty} f(x)=A$,$\forall \varepsilon>0, \exists X>0$, 当 $|x|>X$ 时, 恒有 $|f(x)-A|<\varepsilon$。
	\end{definition}
	
	\begin{theorem}
		\begin{equation}
			\begin{split}
				&\lim _{x\rightarrow \infty}f(x)=A \\
				\Leftrightarrow 
				&\lim _{x\rightarrow + \infty}f(x)
				=\lim _{x\rightarrow - \infty}f(x)=A
			\end{split}
		\end{equation}
	\end{theorem}


	反过来说,当$\lim _{x \rightarrow -\infty}f(x)!=\lim _{x \rightarrow}f(x)$时,我们称$\lim_{x \rightarrow \infty}$不存在。
	同时,在数列中,若$n\rightarrow \infty$,我们约定为$n\rightarrow +\infty$。在数列中,
	$$\lim _{n\rightarrow \infty}f(n) \Rightarrow \lim _{x\rightarrow +\infty}f(x)$$
	该式无法反推:
	对于 $f(x)=\sin{x\pi}$,数列$\lim _{n\rightarrow\infty}f(n)=0$,
	但是函数$f(n)$的极限不存在,这是由于数列只取$1,2,...n$等正整数。
	
	\section{自变量趋向于有限值时}
	\begin{definition}
		$\lim _{x\rightarrow x_0}f(x)=A$,$\forall \varepsilon > 0$,
		$\exists \delta > 0$,当$0<|x-x_0|<\delta$时,恒有$|f(x)-A|<\varepsilon$。
	\end{definition}
	
	注:
	\begin{enumerate}
		\item $\varepsilon$具有任意性,$\delta$具有存在性
		\item 定义2在几何上的意义为,对于$x \in (x_0-\delta,x_0+\delta)$,
		存在$\varepsilon$,使得$y=f(x)\in (A-\varepsilon,A+\varepsilon)$。
		\item $x\rightarrow x_0$,$x!=x_0$。也就是说,$\lim _{x\rightarrow x_0}$ 与 $f(x_0)$无关。
		\item $0<|x-x_0|<\delta$,$|f(x)-A|<\varepsilon$,即$f(x)=A$可以成立。这里的微小定义会引出一些问题。
		
	\end{enumerate}
	\section{例题}
		\begin{enumerate}
			\item 
			由于$\lim_{x\rightarrow 0} \frac{\sin x}{x}=1$,是否可以得到
			$$
			\lim _{x\rightarrow 0}
			\frac{\sin(x\sin {\frac{1}{x}})}
			{x\sin{\frac{1}{x}}}=0
			$$
			
			解:$
			\lim _{x\rightarrow 0}
			\frac{\sin(x\sin {\frac{1}{x}})}
			{x\sin{\frac{1}{x}}}=0
			$不存在。	
			令$xsin \frac{1}{x}=t$,则
			\begin{equation}
				\begin{split}
					&\lim _{x\rightarrow 0}
					\frac{\sin(x\sin {\frac{1}{x}})}
					{x\sin{\frac{1}{x}}}\\
					=&\lim _{x\rightarrow 0}
					\frac{\sin t}{t}
				\end{split}
			\end{equation}
			
			由于$x\rightarrow 0$时,存在$x_0$使得$t=0$
			,则$x\rightarrow 0\neq \sin{\frac{1}{x} \rightarrow 0}$,因此极限不存在。
			
			\item 
			
			根据函数极限的定义证明:函数$f(x)$ 当$x\rightarrow x_0$ 时极限储存在的充分必要条件是左极限、右极限各自存在并且相等。
			
			证:
			\begin{itemize}
				\item [必要性] 若$\lim _{x\rightarrow x_0}f(x)=A$,$\forall \varepsilon > 0, \exists \delta > 0$,当$0<|x-x_0|<\delta$时,有$|f(x)-A|<\varepsilon$.
				
				当$0<x-x_0<\delta$时,$|f(x)-A|<\varepsilon$,即为$\lim _{x \rightarrow x_0^+}=A$;
				
				当$0<x_0-x<\delta$时,$|f(x)-A|<\varepsilon$,即为$\lim _{x \rightarrow x_0^-}=A$;
				
				
				\item [充分性] 反过来即可。
			\end{itemize} 
			
			\item 给出$x\rightarrow \infty$时,函数的极限的局部有界性定理,并加以证明。
			
			答:若$\lim _{x\rightarrow \infty}f(x)=A$,$\exists$常数$ X > 0,M>0$,使得$|x|>X$时,$|f(x)|<=M$.
			
			\begin{equation}
				\begin{split}
					&\lim _{x\rightarrow \infty} f(x)=A \\
					\Rightarrow 
					&\exists \delta>0,|f(x)-A|<\varepsilon
				\end{split}
			\end{equation}
		令$\varepsilon=1$,则
			\begin{equation}
				\begin{split}
					&|f(x)|-|A|<=|f(x)-A|<\varepsilon=1\\
					\Rightarrow
					&|f(x)|<1+|A|
				\end{split}
			\end{equation}
			令$M=1+|A|$,$N=\delta$,则$\exists M,N>0$,在$x>|X|$时,使得$|f(x)|<M$。
			
			\item  (这题是常用极限,但是我认为武证明的比书好)
			
			证明:$\lim_{n\rightarrow \infty}(1+\frac{1}{n})^{n}$存在极限
			
			有均值不等式:
			$$
			a_1\cdot a_2\cdot...\cdot a_n \leq (
			\frac{a_1+a_2+...+a_n}{n})^n
			$$
			可得:
			\begin{equation}
				\begin{split}
					&(1+\frac{1}{n})^n \\
					=&(1+\frac{1}{n})^n*1 
					\leq\frac{(1+n+1)}{n+1}
					=(\frac{1}{n+1})^{n+1}
				\end{split}
			\end{equation}
			原式为单调增序列,为证数列存在极限,只需证明有界:
			\begin{equation}
				\begin{split}
					&\frac{1}{4}\cdot(1+\frac{1}{n})^n \\
					=&(1+\frac{1}{n})^n\cdot \frac{1}{2} \cdot \frac{1}{2}
					\leq\frac{(1+n+1)}{n+2}
					=1\\
					&(1+\frac{1}{n})^n\leq 4
				\end{split}
			\end{equation}
		
			\item 求$\lim_{n\rightarrow \infty} [\frac{1}{1*2}+\frac{1}{2*3}+...+\frac{1}{n(n+1)}]^2$.
			
			解:
			\begin{equation}
				\begin{split}
					&\lim [\frac{1}{1*2}+\frac{1}{2*3}+...+\frac{1}{n(n+1)}\\
					=&(1-\frac{1}{2})+(\frac{1}{2}-\frac{1}{3})+...+
					(\frac{1}{n}-\frac{1}{n+1})\\
					=&1-\frac{1}{n+1}, \\										
					\text{原式}=&\lim_{n\rightarrow \infty} (1-\frac{1}{n+1})^n \\
					=&\lim_{n\rightarrow \infty} [(1+(-\frac{1}{n+1}))^{-(n+1)}]
					^\frac{n}{-n-1} \\
					=&e^{-1}.
				\end{split}
			\end{equation}
		\end{enumerate}
\end{document}